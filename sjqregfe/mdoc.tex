%% Template for Stata-Journal Quarto manuscript

%% Main

% main.tex - a driver for your Stata Journal insert
% This file should only be changed according to the AUTHOR notes below.
% The Stata Press document class

\documentclass[bib]{statapress}
% Page dimensions
\usepackage[crop,newcenter,frame]{pagedims}
% The Stata Journal styles
\usepackage{sj}
% Stata Log listings and useful macros
\usepackage{stata}

% Encapsulated PostScript figures
\usepackage{epsfig}
% Shadow package to render technical note figure
\usepackage{shadow}
\usepackage{amsmath,amssymb} 
% EDITORS: volume number, issue number, month, and year

%CodeThis is for Executed code. But may not be necessary
\usepackage{color}
\usepackage{fancyvrb}
\newcommand{\VerbBar}{|}
\newcommand{\VERB}{\Verb[commandchars=\\\{\}]}
\DefineVerbatimEnvironment{Highlighting}{Verbatim}{commandchars=\\\{\}}
% Add ',fontsize=\small' for more characters per line
\usepackage{framed}
\definecolor{shadecolor}{RGB}{255,255,255}
\newenvironment{Shaded}{\begin{snugshade}}{\end{snugshade}}
\newcommand{\KeywordTok}[1]{\textcolor[rgb]{0.00,0.0,0.0}{#1}}
\newcommand{\NormalTok}[1]{\textcolor[rgb]{0.00,0.0,0.0}{#1}}

%% Ref

 

\sjsetissue{vv}{ii}{mm}{yyyy}

%%%%%%%%%%%%%%%%%%%%%%%%%%%%%%%%%%%%%%%%%%%%%%%%%%%%%%%%%%%%%%%%%%%%%%%%%%%%%%%


\providecommand{\tightlist}{%
  \setlength{\itemsep}{0pt}\setlength{\parskip}{0pt}}\usepackage{longtable,booktabs,array}
\usepackage{calc} % for calculating minipage widths
% Correct order of tables after \paragraph or \subparagraph
\usepackage{etoolbox}
\makeatletter
\patchcmd\longtable{\par}{\if@noskipsec\mbox{}\fi\par}{}{}
\makeatother
% Allow footnotes in longtable head/foot
\IfFileExists{footnotehyper.sty}{\usepackage{footnotehyper}}{\usepackage{footnote}}
\makesavenoteenv{longtable}
\usepackage{graphicx}
\makeatletter
\def\maxwidth{\ifdim\Gin@nat@width>\linewidth\linewidth\else\Gin@nat@width\fi}
\def\maxheight{\ifdim\Gin@nat@height>\textheight\textheight\else\Gin@nat@height\fi}
\makeatother
% Scale images if necessary, so that they will not overflow the page
% margins by default, and it is still possible to overwrite the defaults
% using explicit options in \includegraphics[width, height, ...]{}
\setkeys{Gin}{width=\maxwidth,height=\maxheight,keepaspectratio}
% Set default figure placement to htbp
\makeatletter
\def\fps@figure{htbp}
\makeatother

\makeatletter
\@ifpackageloaded{caption}{}{\usepackage{caption}}
\AtBeginDocument{%
\ifdefined\contentsname
  \renewcommand*\contentsname{Table of contents}
\else
  \newcommand\contentsname{Table of contents}
\fi
\ifdefined\listfigurename
  \renewcommand*\listfigurename{List of Figures}
\else
  \newcommand\listfigurename{List of Figures}
\fi
\ifdefined\listtablename
  \renewcommand*\listtablename{List of Tables}
\else
  \newcommand\listtablename{List of Tables}
\fi
\ifdefined\figurename
  \renewcommand*\figurename{Figure}
\else
  \newcommand\figurename{Figure}
\fi
\ifdefined\tablename
  \renewcommand*\tablename{Table}
\else
  \newcommand\tablename{Table}
\fi
}
\@ifpackageloaded{float}{}{\usepackage{float}}
\floatstyle{ruled}
\@ifundefined{c@chapter}{\newfloat{codelisting}{h}{lop}}{\newfloat{codelisting}{h}{lop}[chapter]}
\floatname{codelisting}{Listing}
\newcommand*\listoflistings{\listof{codelisting}{List of Listings}}
\makeatother
\makeatletter
\makeatother
\makeatletter
\@ifpackageloaded{caption}{}{\usepackage{caption}}
\@ifpackageloaded{subcaption}{}{\usepackage{subcaption}}
\makeatother
\begin{document}

%% AUTHOR:  Include your article here.

%% TITLE

 
\title[Short toc Here]{Estimation of Quantile Regressions with Multiple
Fixed Effects}


\makeatletter

\inserttype[st0001]{article}
\author{Rios-Avila, Siles, Canavire-Bacarreza}{
Fernando Rios-Avila\\
Levy Economics Institute\\Annandale-on-Hudson, NY\\
\href{mailto:friosavi@levy.org}{friosavi@levy.org}
\and 
Leonardo Siles\\
Universidad de Chile\\Santiago, Chile\\
\href{mailto:lsiles@fen.uchile.cl}{lsiles@fen.uchile.cl}
\and 
Gustavo Canavire-Bacarreza\\
The World Bank\\Washington, DC\\
\href{mailto:gcanavire@worldbank.org}{gcanavire@worldbank.org}
}
 
 
 

\maketitle

\begin{abstract}

This is an example of StataJ article made by me

\keywords{\inserttag, Stata, LaTeX, Quarto, StataJ}
\end{abstract}

\section{Introduction}\label{sec-intro}

Quantile regression, introduced by \citet{koenker1978}, has become an
important tool in economic analysis, allowing to examine how the
relationship between the dependent and independent variables varies
across different points of the conditional distribution of the outcome.
While ordinary least squares focuses on analyzing the conditional mean,
quantile regression provides a more comprehensive view of how covariates
impact the entire conditional distribution of the dependent variable.
This can reveal heterogeneous effects that may be otherwise overlooked
when analyzing the conditional mean.

A relatively recent development in the literature has focused on
extending quantile regression analysis in a panel data setting to
account for unobserved, but time fixed heterogeneity. This is
particularly important in empirical research, where unobserved
heterogeneity can bias estimates of the effects of interest. However, as
it is common in the estimation of non-linear models with fixed effects,
introducing fixed effects in quantile regression models poses several
challenges. On the one hand, the simple inclusion of fixed effects can
lead to an incidental parameter problem, which can bias estimates of the
quantile coefficients \citep{neymanscott1948, lancaster2000}. On the
other hand, the computational complexity of estimating quantile
regression models with fixed effects can be prohibitive, particularly
for large datasets with multiple high-dimensional fixed effects. While
many strategies have been proposed for estimating this type of model
(see \citet{galvao2017quantile} for a review), none has become standard
due to restrictive assumptions regarding the inclusion of fixed effects
and the computational complexity.

In spite of the growing interest in estimating quantile regression
models with fixed effects in applied research, particularly in the
fields of labor economics, health economics, and public policy, among
others, there are few commands that allow the estimation of such models.
In Stata, there are three main built-in commands available for
estimating quantile regression \texttt{qreg}, \texttt{ivqregress}, and
\texttt{bayes:\ qreg}, and none of them allow for the inclusion of fixed
effects, other than using the dummy variable approach. From the
community-contributed commands, there is \texttt{xtqreg}, which
implements a quantile regression model with fixed effects based on the
method of moments proposed by \citet{mss2019}, and more recently
\texttt{xtmdqr} which implements a minimum distance estimation of
quantile regression models with fixed effects described in
\citet{melly2023}. In both cases, these command are constrained to a
single set of fixed effects.\footnote{There are other
  community-contributed commands like \texttt{xtrifreg},
  \texttt{rifhdfe}, \texttt{qregpd}, \texttt{rqr} among others that
  allow for the estimation of quantile regression models, but do not
  estimate conditional quantile regressions, but instead focus on
  unconditional quantile regressions, or quantile treatment effects.}

To address this, in this paper we introduce two Stata commands for
estimating quantile regressions with multiple fixed effects:
\texttt{mmqreg} and \texttt{qregfe}. The first command \texttt{mmqreg}
is an extension of the method of moments quantile regression estimator
proposed by \citet{mss2019}. The second \texttt{qregfe}, implements
three other approaches: an implementation of a correlated random effects
estimator based on \citet{abrevaya2008}, \citet{wooldridge2019} and
\citet[Ch12.10.3]{wooldridge2010}; the estimator proposed by
\citet{canay2011}, and a proposed modification of this approach. In
addition, we also present an auxiliary command \texttt{qregplot} for the
visualization of the quantile regression models.

Both commands offer the advantage of allowing for the estimation of
conditional quantile regressions while controlling for multiple fixed
effects. First, they leverage over existing Stata commands, as well as
other community-contributed commands, to allow users to estimate
quantile regression models and their standard errors under different
assumptions. Second, they reduce the impact of the incidental parameters
problem under different assumptions regarding the data generating
process. In terms of standard errors, \texttt{mmqreg} allows for the
estimation of analytical standard errors (see \citet{mss2019} and
\citet{riosavila2024}), whereas \texttt{qregfe} emphasizes the use of
bootstrap standard errors. Finally, both commands are designed to be
user-friendly, allowing for the estimation of quantile regression models
with fixed effects in a single line of code.

The remainder of the paper is organized as follows. Section 2 reviews
the methodological framework for quantile regression. Section 3
describes the methods and formulas used by \texttt{mmqreg} and
\texttt{qregfe} commands. Section 4 introduces the commands, along with
a brief description of their syntax and options. Section 5 introduces an
auxiliary command for the visualization of quantile regression models.
Section 6 provides an empirical applications demonstrating their use.
Section 7 concludes.

\section{The Basics}\label{sec-basics}

While standard regression techniques allow the researcher to determine
the effect of covariates \(X\) on the expected value of a response
variable \(Y\), regression quantiles are capable of characterizing how
changes in \(X\) affect the entire distribution of \(Y\). In applied
research, this may be useful when one seeks to determine how does a
given percentile of the distribution of interest respond to changing
values of explanatory variables. For example, consider the effects of
smoking on birth weight, as \citep{abrevaya2008} studied in the context
of quantile regression. Instead of fixing an \emph{unconditional
threshold} so as to divide the sample in low birth weights and non-low
birth weights, the authors followed a \emph{conditional quantile}
approach that enabled them to study the effects of birth inputs
(e.g.~smoking) in different parts of the birth weight distribution
(\citep{abrevaya2008}).

In its more common setting, quantile regressions are assumed to follow a
linear form:

\begin{equation}\phantomsection\label{eq-qr}{
Q_{\tau}(Y | X) =  X \beta(\tau)
}\end{equation}

Where \(Q_{\tau}(Y | X)\) is the \(\tau\)-th quantile of \(Y\)
conditional on

Such that, while the parameters of interest change across quantiles, the
effects of covariates \(X\) on the \(\tau\)-th quantile of \(Y\) are
assumed to be linear.

In addition to heterogeneous effects of explanatory variables on the
outcome of interest, the researcher might be interested as well in
properly addressing the problem of unobserved heterogeneity. Returning
to the example of birth weights, it may be the case that unobservable
characteristics of the mother are correlated with both the birth outcome
and birth inputs, such as her smoking status during pregnancy or whether
she attended prenatal care visits (\citep{abrevaya2008}). In this case,
a panel data approach may be useful for obtaining causal estimates of
the effects of explanatory variables on quantiles of \(Y\).

At the same time, panel data models with unobserved time-invariant
effects depend upon demeaning techniques to eliminate such effects and
yield consistent estimators of the parameters of interest. However,
differentiating out individual effects is not a feasible approach within
the context of regression quantiles, the reason being that quantiles are
\emph{not} linear operators. Many authors, for example
\citep{abrevaya2008}, \citep{canay2011} or \citep{mss2019} have stated
that this key limitation has prevented further progress in the
estimation of regression quantiles with fixed effects.

In this paper, we review four estimators for regression quantiles in the
presence of fixed effects. All of them are relatively simple to
implement with our provided commands for Stata (see sections
\hyperref[sec:cre]{3}, \hyperref[sec:canay]{4} and
\hyperref[sec:mmqreg]{5} below). However, this easiness comes at the
cost of making important assumptions regarding, among others, the data
generating process (DGP) and the effect of unobserved heterogeneity
\(\alpha_i\) on the distribution of \(Y\). The following sections
describe each of the estimators, their key assumptions and provide
examples for their implementation in Stata.

\section{Correlated Random Effects}\label{sec:cre}

This estimator, described by \citep{wooldridge2010}, rests upon the
Mundlak repsesentation of a Correlated Random Effects (CRE) model for
the estimation of regression quantiles with panel data. The underlying
DGP for the outcome of interest \(Y_{it}\) is:

\[\label{eq:dgp_cre}
    Y_{it} = X_{it}' \beta + \alpha_i + u_{it}\]

where \(\alpha_i\) is the time-invariant individual effect and can be
described in terms of \(X_i \equiv (X_{i1},...,X_{iT})\) as follows:

\[\label{eq:mundlak_rep}
    \alpha_i = \psi + \Bar{X_i}' \lambda + v_i\]

where \(\Bar{X_i}\) denotes the time average of characteristics across
individuals. Then, the conditional \(\tau\)-th quantile of \(Y_{it}\)
is:

\[\label{eq:qr_cre}
    Q_{\tau}(Y_{it} | X_i) = \psi + X_{it}' \beta + \Bar{X_i}' \lambda + Q_{\tau}(a_{it} | X_i)\]

where \(a_{it} \equiv v_i + u_{it}\) is the composite error. The CRE
estimator using the Mundlak representation imposes the following key
assumption:

\[\begin{gathered}
    \text{CRE1.\quad $a_{it}$ is independent of $X_i$.}\label{eq:cre1}
    %\text{CRE1.}\quad v_i\; \text{is independent of}\; X_i \label{eq:cre1} \\
    %\text{CRE2.}\quad Q_{\tau}(u_{it} | X_i, v_i) = Q_{\tau}(u_{it} | X_{it}) \label{eq:cre2}
\end{gathered}\]

Stringent as may be, this assumption allows to consistently estimate
\(\beta\) and \(\lambda\) from a pooled quantile regression of
\(Y_{it}\) on \(X_{it}\) and \(\Bar{X_{i}}\). As \citep{wooldridge2010}
remarks, CRE1 amounts to assuming that quantile functions are parallel
or, in other words, that by imposing a relationship between \(X_{i}\)
and \(\alpha_i\) of the form displayed in equation
\hyperref[eq:mundlak_rep]{{[}eq:mundlak\_rep{]}} we can only estimate
the location shifts of quantile regressions. As it will be shown below,
this is a feature the CRE estimator shares with Canay's method (see
section \hyperref[sec:canay]{4}) for estimating regression quantiles in
a panel data context. Both estimators are constrained in the sense that
only location effects on quantile functions are accounted for, whereas
the MMQREG method (see section \hyperref[sec:mmqreg]{5}) has the ability
to estimate quantile coefficients that take into account both location
and scale effects.

\citep{abrevaya2008} develop an alternative CRE estimator for regression
quantiles with panel data drawing on the model of
\citep{chamberlain1982}. However, its implementation is severely
restricted because Abrevaya and Dahl's estimator regresses \(Y_{it}\) on
\(X_{it}\) and \(X_{i}\) for a given \(\tau\). Even if we have a
balanced panel available for our application, we will have to deal with
the large quantity of regressors included in the equation
{[}\(K(1+T)\){]} and its corresponding loss of degrees of freedom for
statistical inference.

Introduction of multiple fixed effects in this setting is
straightforward. First, modify equation
\hyperref[eq:dgp_cre]{{[}eq:dgp\_cre{]}} so as to include two or more
sets of fixed effects (\(\alpha^1, \alpha^2, ..., \alpha^k\)) and
proceed by computing the averages of explanatory variables \(X\) with
respect to the variable identifying the fixed effect. Finally, run a
regression of \(Y\) on the set of original control variables and the
averages just calculated. In terms of the quantile function, equation
\hyperref[eq:qr_cre]{{[}eq:qr\_cre{]}} requires to be modified
accordingly:

\[\label{eq:qr_cre_twoway}
    Q_{\tau}(Y_{it} | X_i) = \Psi + X_{it}' \beta + \Bar{X^1_i}' \lambda^1 + \Bar{X^2_i}' \lambda^2 + Q_{\tau}(a_{it} | X_i)\]

For the case of twoway fixed effects and where
\(\Psi \equiv \psi^1 + \psi^2\).

Implementation of the CRE estimator for quantile regression in Stata can
be done with the command \texttt{cre}, which is actually a prefix for
the estimation of CRE models. Option
\texttt{{[}abs(}\emph{\texttt{varlist}}\texttt{){]}} is used to provide
the command with variables identifying groups of observations that will
be subsequently used to estimate the fixed effects. By default,
singleton groups are dropped from the estimation. Now, we present an
application of the CRE estimator for regression quantiles with multiple
fixed effects using the aforementioned command. Note that the same
dataset will be used for all estimation methods surveyed in this paper,
so it will be worth to outline the main features of the data before
proceeding to display, comment and compare the estimation results.

We use the panel from \citep{persson2007} who study the effect of
alternative types of government on its overall spending in the last year
of the legislature. The authors consider both coalition and single-party
as government types, leaving out from consideration minority
governments. Here, we will only focus on the relationship a coalition
government has with its overall spending and expand \citep{persson2007}
results by estimating the .25, .5 and .75 quantiles of the government
spending distribution.

Variables used for estimation include central government spending as
percentage of GDP in the last year of legislature (\emph{last\_exp}), an
indicator variable for coalition governments (\emph{coalition}),
population size in log scale (\emph{lpop}), the percentage of population
over 65 years of age (\emph{prop65}), the log-deviation of output from
the country trend (\emph{ygap}) and the length (in years) of the
legislature (\emph{length}). Both country and period fixed effects are
included in our estimations, in the same fashion as \citep{persson2007}.

\phantomsection\label{tab:cre_results}
\begin{longtable}[]{@{}lll@{}}
\caption{Government spending and coalition government - CRE
Estimator}\tabularnewline
\toprule\noalign{}
Column 1 & Column 2 & Column 3 \\
\midrule\noalign{}
\endfirsthead
\toprule\noalign{}
Column 1 & Column 2 & Column 3 \\
\midrule\noalign{}
\endhead
\bottomrule\noalign{}
\endlastfoot
Data 1 & Data 2 & Data 3 \\
Data 4 & Data 5 & Data 6 \\
\end{longtable}

\phantomsection\label{tab:cre_results}{}

Estimation results from \citep{persson2007} dataset using the CRE
estimator for quantile regression are displayed in table
\hyperref[tab:cre_results]{1} and show heterogeneous effects of a
permanent switch from a single-party to a coalition government on
government spending. The largest effect is obtained at the median of the
distribution, and the estimated coefficient is close to the one
\citep{persson2007} computed using the within estimator (2.36). At the
lower and upper quartile of the distribution we still find positive
coefficients, which support the theoretical predictions of the model
developed by the same authors. However, none of the coefficients for
coalition across quantiles are statistically different from zero. In the
context of CRE estimation, this fact may be partly due to the loss of
degrees of freeedom implied by the addition of 12 extra regressors to
the equation. In general, CRE will add \(W \times K\) explanatory
variables, with \(W\) being the number of fixed effects we are
accounting for.

\section{Canay (2011) Estimator}\label{sec:canay}

Canay's estimator key assumption is that fixed effects are merely
treated as location-shifters. That is, unobserved individual effects
only change the distribution of the response variable in a parallel
fashion. While this may be a rather strong assumption, it helps to
consistently estimate the parameters of interest using a simple
transformation of the data as detailed below \citep{canay2011}.

The underlying DGP of \(Y_{it}\) is:

\[\label{eq:dgp_canay}
    Y_{it} = X_{it}' \beta(u_{it}) + \alpha_i\]

To consistently estimate the \(\beta_{\tau}\) from the data, the
following assumptions are in place:

\[\begin{gathered}
    \text{CANAY1. Conditional on $X_{it} = x$ the random variables $X_{it}$ and $\alpha$ are independent.} \\
    \text{CANAY2. $U_{it}$ is independent of $X_i$ and $\alpha_i$, with $U_{it} \sim U[0,1]$} \\
    \text{CANAY3. $\beta_{\mu} \equiv E[\beta(U_{it})]$. That is, $\beta_{\mu}$ exists.}
\end{gathered}\]

Start defining the following equation, which is simply a restatement of
equation \hyperref[eq:dgp_canay]{{[}eq:dgp\_canay{]}}, as follows:

\[\label{eq:canay_2.3}
Y_{it} = X_{it}' \beta(\tau) + \alpha_i + e_{it}(\tau)\]

Where \(e_{it}(\tau) \equiv X_{it}'[\beta(U_{it} - \beta(\tau)]\). Then,
due to assumption CANAY2, in the above equation only \(\beta(\tau)\) and
\(e_{it}(\tau)\) depend on \(\tau\). Next, write a conditional mean
equation for \(Y_{it}\) as the following:

\[\label{eq:canay_meaneq}
    Y_{it} = X_{it}' \beta_{\mu} + \alpha_i + u_{it}\]

Where \(u_{it} \equiv X_{it}' [\beta(U_{it}) - \beta_{\mu}]\).
Assumption CANAY3 allows us to write this kind of mean equation. Then,
the two-step estimator due to \citep{canay2011} is defined next:

\emph{Step 1.} Let \(\hat{\beta}_\mu\) be a \(\sqrt{nT}\)-consistent
estimator of \(\beta_{\mu}\). Define
\(\hat{\alpha}_i \equiv \mathbb{E}_T[Y_{it} - X'_{it}\hat{\beta}_{\mu}]\).

\emph{Step 2.} Let \(\hat{Y}_{it} = Y_{it} - \hat{\alpha}_i\) and define
the two-step estimator as:

\[\label{eq:qreg_canay}
    \hat{\beta}(\tau) \equiv \min_{\beta \in \beta} \mathbb{E}_{nT} [\rho(\hat{Y}_{it} - X_{it}' \beta)]\]

Where \(\rho(\cdot)\) is the check function \citep{koenker1978},
\(\mathbb{E}_{T}(\cdot) \equiv T^{-1} \sum_{t=1}^{T} (\cdot)\) and
\(\mathbb{E}_{nT}(\cdot) \equiv (nT)^{-1} \sum_{t=1}^{T} \sum_{i=1}^n(\cdot)\).
\citep{canay2011} defines this estimator as ``consistent and
asymptotically normal". This stated, Canay's estimator stands out for
its simplicity: Step 1 may be conveniently restated as obtaining the
known within estimator of \(\beta\) and then subtracting the
time-average residuals from the dependent variable. Next, using this
simple transformation of the data, Step 2 tells us to obtain regression
quantiles of \(\hat{Y}_{it}\) that are guaranteed to be consistent, as
long as key assumptions hold.

It is important to remark that consistency of Canay's estimator is
obtained when \(T \rightarrow \infty\) \citep{canay2011}. Thus, the
estimator at hand may perform poorly in contexts of panels with few time
periods. In addition to this, the key assumption of the approach relies
on the fact that the \(\alpha_i\)'s are pure location shifts: in other
words, that the unobservable time-invariant characteristics grouped in
\(\alpha_i\) have coefficients that are constant across \(\tau\)
\citep{canay2011}. As \citep{mss2019} point out using simulated data,
the estimator's performance deteriorates as the DGP features fixed
effects that alter the entire distribution of interest, not only its
location.

Regarding statistical inference, \citep{canay2011} provides a simple
algorithm to obtain bootstrapped standard errors. The key in this
procedure is to compute the two step estimator for each bootstrap
sample, avoiding the mistake of just sampling from the transformed data
\(\hat{Y}_{it}\). As \citep{canay2011} remarks, the algorithm should be
such that it computes the first step estimators \(\hat{\beta}_{\mu}\)
and \(\hat{\alpha}_i\) for each repetition.

Finally, we propose a Modified Canay estimator in which, instead of
subtracting out the estimated fixed effects \(\hat{\alpha}_i\) from the
dependent variable \(Y_{it}\), they are included as regressors in
equation \hyperref[eq:qreg_canay]{{[}eq:qreg\_canay{]}}. In stark
contrast to the CRE estimator (see section \hyperref[sec:cre]{3}), the
Modified Canay only adds \(W\) variables to the estimation, which makes
it preferable in contexts of a small sample size when our goal is to
estimate only location shift effects. However, we must be aware of the
fact that the requirement of \(T \rightarrow \infty\) for consistency
may not be met, especially for short panels.

Describe the implementation in Stata here.

\begin{longtable}[]{@{}lll@{}}
\caption{as}\tabularnewline
\toprule\noalign{}
Column 1 & Column 2 & Column 3 \\
\midrule\noalign{}
\endfirsthead
\toprule\noalign{}
Column 1 & Column 2 & Column 3 \\
\midrule\noalign{}
\endhead
\bottomrule\noalign{}
\endlastfoot
Data 1 & Data 2 & Data 3 \\
Data 4 & Data 5 & Data 6 \\
\end{longtable}

Tables \hyperref[tab:canay_results]{2} and
\hyperref[tab:mcanay_results]{3} display the results of Canay and
Modified Canay estimations using the \citep{persson2007} dataset of
government expenditures. Once again, the median coefficients are close
to the mean coefficient from the original paper, with the Canay
estimator delivering the closest coefficient to 2.36. Even further,
statistical significance is achieved for the median coefficient in table
\hyperref[tab:canay_results]{2}, strengthening our reasoning on degrees
of freedom above. Otherwise, we still obtain positive coefficients of
the coalition variable at the 25th and 75th percentile of government
spending.

It is also worth stating the increasing monotonic pattern of quantile
coefficients of coalition in table \hyperref[tab:mcanay_results]{3},
conveying us that the effect of a permanent switch from single-party to
coalition governments increases as we move toward greater government
spending. Finally, note that even after explicitly accounting for
country and period of legislature fixed effects in the Modified Canay
estimator, the median coefficient remains stable, although it is no
longer significant.

\begin{longtable}[]{@{}lll@{}}
\toprule\noalign{}
Column 1 & Column 2 & Column 3 \\
\midrule\noalign{}
\endhead
\bottomrule\noalign{}
\endlastfoot
Data 1 & Data 2 & Data 3 \\
Data 4 & Data 5 & Data 6 \\
\end{longtable}

\phantomsection\label{tab:mcanay_results}{}

\section{Method of Moments Quantile Regression}\label{sec:mmqreg}

This approach of estimating regression quantiles with multiple fixed
effects distinguishes itself from the other estimators reviewed above in
the sense that \citep{mss2019} introduce location-scale effects of fixed
effects upon the distribution of interest. Compared to Canay's
estimator, the Method of Moments Quantile Regression (MMQREG) not only
allows the \(\alpha_i\)'s to affect \(Y_{it}\) through location shifts,
but rather MMQREG is able to identify the scale shifts that alter
different points of the distribution belonging to \(Y\).

We begin by defining the DGP of the location scale model:

\begin{equation}\phantomsection\label{eq-dgp_mmqreg}{Y_{it} = \alpha_i + X_{it}' \beta + (\delta_i + X_{it}' \gamma) u_{it}
}\end{equation}

Where parameters \(\alpha_i\) and \(\delta_i\) capture the individual
fixed effects. Note that, compared to equation
\hyperref[eq:dgp_canay]{{[}eq:dgp\_canay{]}}, the fixed effects not only
enter the model in an additive fashion, instead they also have a
multiplicative effect upon the error term. In addition, the \(U_{it}\)
are i.i.d. across \(i\) and \(t\), statistically independent of
\(X_{it}\) and satisfy \(E(U) = 0\) and \(E(|U|) = 1\) both of which
normalize the random variable.

Our location scale model in equation
\hyperref[eq:dgp_mmqreg]{{[}eq:dgp\_mmqreg{]}} implies that:

\[\label{eq:mmqreg_quantile}
    Q_{\tau}(Y_{it}|X_i) = [\alpha_i + \delta_i q(\tau)] + X_{it}' \beta + X_{it}' \gamma q(\tau)\]

Where the scalar coefficient
\(\alpha_i(\tau) \equiv \alpha_i + \delta_i q(\tau)\) is the
quantile-\(\tau\) fixed effect for individual \(i\), which represent how
time-invariant variables have \emph{different impacts on different
regions} of the conditional distribution of \(Y_{it}\). However, our
real interest is in the regression quantile coefficients:

\[\label{eq:rqcoefficients_mmqreg}
    \beta_{\tau} = \beta + q(\tau) \gamma\]

Which are simply a linear combination of the location coefficients
(\(\beta\)) and the scale coefficients (\(\gamma\)), where the second
vector of coefficients is weighted by the value of the \(\tau\)-th
quantile of the variable of interest \(Y_{it}\). \citep{mss2019} develop
the following algorithm for implementing the MMQREG estimator:

\begin{enumerate}
\def\labelenumi{\arabic{enumi}.}
\item
  Obtain \(\hat{\beta}_k\) by regressing time-demeaned \(Y_{it}\) on
  time-demeaned controls \(X_{it}\), i.e.~obtain \(\hat{\beta}\) by the
  within estimator.
\item
  Estimate the \(\hat{\alpha}_i\)'s and calculate the residuals
  \(\hat{R}_{it} = Y_{it} - \hat{\alpha}_i - X_{it}' \hat{beta}\).
\item
  Obtain \(\hat{\gamma}_k\) by the within estimator using
  \(|\hat{R}_{it}|\) as the dependent variable.
\item
  Estimate \(\hat{\delta}_i\) by taking the time average of
  \(|\hat{R}_{it}| - X_{it}' \hat{\gamma}\).
\item
  Estimate \(q(\tau)\) by \(\hat{q}\), which corresponds to the
  regression quantile of standardized residuals
  {[}\(\hat{R}_{it}/(\hat{\gamma}_{i} + X_{it}' \hat{\gamma})]\) upon an
  intercept term.
\end{enumerate}

Note that Steps 1 and 2 from the MMQREG algorithm are the same as those
performed in Canay's estimator. However, the location-scale model used
in \citep{mss2019} adds three additional steps that are required to
estimate \(\hat{\gamma}\) and \(\hat{q}\), so that regression quantile
coefficients \(\beta_{\tau}\) are allowed to affect not only the
location of the distribution, but also its shape.

Add notes on the assumptions of MMQREG here \emph{if} necessary.

Statistical inference can be performed using the asymptotic distribution
of the estimator derived in \citep{mss2019}. Expanding on this
literature, \citep{riosavila2024} propose methods for computing
alternative standard errors using the empirical influence functions of
the estimators. Robust and clustered standard errors can be estimated
following this approach, and are readily available as options in the
\texttt{mmqreg} command in Stata. Another extension of the MMQREG
estimator due to \citep{riosavila2024} is to allow for the inclusion of
multiple fixed effects using an application of the Frisch-Waugh-Lovell
(FWL) theorem to partial out the effect of variables capturing
unobserved heterogeneity from both dependent and explanatory variables.
Likewise, the command \texttt{mmqreg} allows for multiple fixed effects,
as will be shown next.

\begin{longtable}[]{@{}lll@{}}
\caption{Government spending and coalition government - MMQREG
Estimator}\tabularnewline
\toprule\noalign{}
Column 1 & Column 2 & Column 3 \\
\midrule\noalign{}
\endfirsthead
\toprule\noalign{}
Column 1 & Column 2 & Column 3 \\
\midrule\noalign{}
\endhead
\bottomrule\noalign{}
\endlastfoot
Data 1 & Data 2 & Data 3 \\
Data 4 & Data 5 & Data 6 \\
\end{longtable}

\phantomsection\label{tab:mmqreg_results}{}

As we allow for both location and scale shifts in our estimation method,
the monotonic pattern observed in the first row of table
\hyperref[tab:mcanay_results]{3} is reversed. Now, the farther we move
from the median of the distribution in the direction of greater
government spending, the lower is the effect of a permanent change from
single-party to coalition upon spending. The median coefficient, once
again, remains close to the mean coefficient (2.36) under MMQREG
estimation. Results from table \hyperref[tab:mmqreg_results]{4} support
the conclusions from \citep{persson2007} regarding types of government
and its spending during the last year of legislature, expanding the
results to the response of the distribution of spending upon changes in
explanatory variables.

We must also note that statistical inference ---with robust standard
errors displayed in table \hyperref[tab:mmqreg_results]{4}--- for the
\citep{persson2007} dataset show that many coefficients that with the
previous estimators were not statistically different from zero, are now
significant even at the 99\% confidence level. This is due to the
ability to compute robust standard errors for the MMQREG estimator,
contrasting with the CRE and Canay estimators, for which we estimated
the covariance matrix using the bootstrap. Although not displayed here,
our findings are robust to clustering standard errors for country so
that the coefficients of 25th and 50th quantiles corresponding to the
indicator variable for coalition government remain statistically
significant.

\clearpage

\bibliographystyle{sj}
\bibliography{bibliography.bib}


\begin{aboutauthors}

Fernando Rios-Avila is a Research Scholar at the Levy Economics
Institute of Bard College.

Gustavo Canavire-Bacarreza is a Senior Economist at the World Bank.

\end{aboutauthors}

\end{document}
