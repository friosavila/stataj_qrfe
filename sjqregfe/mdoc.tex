%% Template for Stata-Journal Quarto manuscript

%% Main

% main.tex - a driver for your Stata Journal insert
% This file should only be changed according to the AUTHOR notes below.
% The Stata Press document class

\documentclass[bib]{statapress}
% Page dimensions
\usepackage[crop,newcenter,frame]{pagedims}
% The Stata Journal styles
\usepackage{sj}
% Stata Log listings and useful macros
\usepackage{stata}

% Encapsulated PostScript figures
\usepackage{epsfig}
% Shadow package to render technical note figure
\usepackage{shadow}
\usepackage{amsmath,amssymb} 
% EDITORS: volume number, issue number, month, and year

%CodeThis is for Executed code. But may not be necessary
\usepackage{color}
\usepackage{fancyvrb}
\newcommand{\VerbBar}{|}
\newcommand{\VERB}{\Verb[commandchars=\\\{\}]}
\DefineVerbatimEnvironment{Highlighting}{Verbatim}{commandchars=\\\{\}}
% Add ',fontsize=\small' for more characters per line
\usepackage{framed}
\definecolor{shadecolor}{RGB}{255,255,255}
\newenvironment{Shaded}{\begin{snugshade}}{\end{snugshade}}
\newcommand{\KeywordTok}[1]{\textcolor[rgb]{0.00,0.0,0.0}{#1}}
\newcommand{\NormalTok}[1]{\textcolor[rgb]{0.00,0.0,0.0}{#1}}

%% Ref

 

\sjsetissue{vv}{ii}{mm}{yyyy}

%%%%%%%%%%%%%%%%%%%%%%%%%%%%%%%%%%%%%%%%%%%%%%%%%%%%%%%%%%%%%%%%%%%%%%%%%%%%%%%


\providecommand{\tightlist}{%
  \setlength{\itemsep}{0pt}\setlength{\parskip}{0pt}}\usepackage{longtable,booktabs,array}
\usepackage{calc} % for calculating minipage widths
% Correct order of tables after \paragraph or \subparagraph
\usepackage{etoolbox}
\makeatletter
\patchcmd\longtable{\par}{\if@noskipsec\mbox{}\fi\par}{}{}
\makeatother
% Allow footnotes in longtable head/foot
\IfFileExists{footnotehyper.sty}{\usepackage{footnotehyper}}{\usepackage{footnote}}
\makesavenoteenv{longtable}
\usepackage{graphicx}
\makeatletter
\def\maxwidth{\ifdim\Gin@nat@width>\linewidth\linewidth\else\Gin@nat@width\fi}
\def\maxheight{\ifdim\Gin@nat@height>\textheight\textheight\else\Gin@nat@height\fi}
\makeatother
% Scale images if necessary, so that they will not overflow the page
% margins by default, and it is still possible to overwrite the defaults
% using explicit options in \includegraphics[width, height, ...]{}
\setkeys{Gin}{width=\maxwidth,height=\maxheight,keepaspectratio}
% Set default figure placement to htbp
\makeatletter
\def\fps@figure{htbp}
\makeatother

\makeatletter
\@ifpackageloaded{caption}{}{\usepackage{caption}}
\AtBeginDocument{%
\ifdefined\contentsname
  \renewcommand*\contentsname{Table of contents}
\else
  \newcommand\contentsname{Table of contents}
\fi
\ifdefined\listfigurename
  \renewcommand*\listfigurename{List of Figures}
\else
  \newcommand\listfigurename{List of Figures}
\fi
\ifdefined\listtablename
  \renewcommand*\listtablename{List of Tables}
\else
  \newcommand\listtablename{List of Tables}
\fi
\ifdefined\figurename
  \renewcommand*\figurename{Figure}
\else
  \newcommand\figurename{Figure}
\fi
\ifdefined\tablename
  \renewcommand*\tablename{Table}
\else
  \newcommand\tablename{Table}
\fi
}
\@ifpackageloaded{float}{}{\usepackage{float}}
\floatstyle{ruled}
\@ifundefined{c@chapter}{\newfloat{codelisting}{h}{lop}}{\newfloat{codelisting}{h}{lop}[chapter]}
\floatname{codelisting}{Listing}
\newcommand*\listoflistings{\listof{codelisting}{List of Listings}}
\makeatother
\makeatletter
\makeatother
\makeatletter
\@ifpackageloaded{caption}{}{\usepackage{caption}}
\@ifpackageloaded{subcaption}{}{\usepackage{subcaption}}
\makeatother
\begin{document}

%% AUTHOR:  Include your article here.

%% TITLE

 
\title[Short toc Here]{Estimation of Quantile Regressions with Multiple
Fixed Effects}


\makeatletter

\inserttype[st0001]{article}
\author{Rios-Avila, Siles, Canavire-Bacarreza}{
Fernando Rios-Avila\\
Levy Economics Institute\\Annandale-on-Hudson, NY\\
\href{mailto:friosavi@levy.org}{friosavi@levy.org}
\and 
Leonardo Siles\\
Universidad de Chile\\Santiago, Chile\\
\href{mailto:lsiles@fen.uchile.cl}{lsiles@fen.uchile.cl}
\and 
Gustavo Canavire-Bacarreza\\
The World Bank\\Washington, DC\\
\href{mailto:gcanavire@worldbank.org}{gcanavire@worldbank.org}
}
 
 
 

\maketitle

\begin{abstract}

This is an example of StataJ article made by me

\keywords{\inserttag, Stata, LaTeX, Quarto, StataJ}
\end{abstract}

\section{Introduction}\label{sec-intro}

Quantile regression, introduced by \citet{koenker1978}, has become an
important tool in economic analysis, allowing to examine how the
relationship between the dependent and independent variables varies
across different points of the conditional distribution of the outcome.
While ordinary least squares focuses on analyzing the conditional mean,
quantile regression provides a more comprehensive view of how covariates
impact the entire conditional distribution of the dependent variable.
This can reveal heterogeneous effects that may be otherwise overlooked
when analyzing the conditional mean.

A relatively recent development in the literature has focused on
extending quantile regression analysis in a panel data setting to
account for unobserved, but time fixed heterogeneity. This is
particularly important in empirical research, where unobserved
heterogeneity can bias estimates of the effects of interest. However, as
it is common in the estimation of non-linear models with fixed effects,
introducing fixed effects in quantile regression models poses several
challenges. On the one hand, the simple inclusion of fixed effects can
lead to an incidental parameter problem, which can bias estimates of the
quantile coefficients \citep{neymanscott1948, lancaster2000}. On the
other hand, the computational complexity of estimating quantile
regression models with fixed effects can be prohibitive, particularly
for large datasets with multiple high-dimensional fixed effects. While
many strategies have been proposed for estimating this type of model
(see \citet{galvao2017quantile} for a review), none has become standard
due to restrictive assumptions regarding the inclusion of fixed effects
and the computational complexity.

In spite of the growing interest in estimating quantile regression
models with fixed effects in applied research, particularly in the
fields of labor economics, health economics, and public policy, among
others, there are few commands that allow the estimation of such models.
In Stata, there are three main built-in commands available for
estimating quantile regression \texttt{qreg}, \texttt{ivqregress}, and
\texttt{bayes:\ qreg}, and none of them allow for the inclusion of fixed
effects, other than using the dummy variable approach. From the
community-contributed commands, there is \texttt{xtqreg}, which
implements a quantile regression model with fixed effects based on the
method of moments proposed by \citet{mss2019}, and more recently
\texttt{xtmdqr} which implements a minimum distance estimation of
quantile regression models with fixed effects described in
\citet{melly2023}. In both cases, these command are constrained to a
single set of fixed effects.\footnote{There are other
  community-contributed commands like \texttt{xtrifreg},
  \texttt{rifhdfe}, \texttt{qregpd}, \texttt{rqr} among others that
  allow for the estimation of quantile regression models, but do not
  estimate conditional quantile regressions, but instead focus on
  unconditional quantile regressions, or quantile treatment effects.}

To address this, in this paper we introduce two Stata commands for
estimating quantile regressions with multiple fixed effects:
\texttt{mmqreg} and \texttt{qregfe}. The first command \texttt{mmqreg}
is an extension of the method of moments quantile regression estimator
proposed by \citet{mss2019}. The second \texttt{qregfe}, implements
three other approaches: an implementation of a correlated random effects
estimator based on \citet{abrevaya2008}, \citet{wooldridge2019} and
\citet[Ch12.10.3]{wooldridge2010}; the estimator proposed by
\citet{canay2011}, and a proposed modification of this approach. In
addition, we also present an auxiliary command \texttt{qregplot} for the
visualization of the quantile regression models.

Both commands offer the advantage of allowing for the estimation of
conditional quantile regressions while controlling for multiple fixed
effects. First, they leverage over existing Stata commands, as well as
other community-contributed commands, to allow users to estimate
quantile regression models and their standard errors under different
assumptions. Second, they reduce the impact of the incidental parameters
problem depending on the assumptions of the underlying the data
generating process. In terms of standard errors, \texttt{mmqreg} allows
for the estimation of analytical standard errors (see \citet{mss2019}
and \citet{riosavila2024}), whereas \texttt{qregfe} emphasizes the use
of bootstrap standard errors. Finally, both commands are designed to be
user-friendly, allowing for the estimation of quantile regression models
with fixed effects in a single line of code.

The remainder of the paper is organized as follows. Section 2 reviews
the methodological framework for quantile regression. Section 3
describes the methods and formulas used by \texttt{mmqreg} and
\texttt{qregfe} commands. Section 4 introduces the commands, along with
a brief description of their syntax and options. Section 5 introduces an
auxiliary command for the visualization of quantile regression models.
Section 6 provides an empirical applications demonstrating their use.
Section 7 concludes.

\section{The Basics}\label{sec-basics}

Quantile regressions allow researchers to identify the heterogenous
effect covariates could have over the entire conditional distribution of
the dependent variable. Let \(y_i\) be the dependent variable, \(x_i\)
the vector of covariates, and \(0<\tau<1\) is a parameter such that
\(q_\tau(y_i|X)\) identifies the \(\tau th\) quantile of the conditional
distribution of \(y_i|X\). Under the assumption that conditional
quantiles are linear functions of the parameters, the quantile
regression model can be written as:

\begin{equation}\phantomsection\label{eq-qr}{q_\tau(y_i|X)=x_i\beta(\tau)
}\end{equation}

Where \(\beta(\tau)\) is the vector of coefficients that may vary across
\(\tau\) and needs to be estimated, and \(x_i\) is a vector of exogenous
covariates that may include nonlinear functions of underlying variables.
This expression indicates that, conditional on \(X\), the \(\tau\)-th
quantile of \(Y\) can be approximated by a linear function of \(X\).

As explained in \citet{wooldridge2010}, the coefficient of quantile
regression models can be identified by minimizing the following loss
function, with respect to \(\beta(\tau)\):

\begin{equation}\phantomsection\label{eq-qloss}{
\hat\beta(\tau) = \min_{\beta(\tau)} \sum_{i=1}^{n} \rho_\tau \big(y_i-x_i\beta(\tau)\big)
}\end{equation}

Where \(\rho_\tau(u)=u\big(\tau-I(u<0)\big)\) is the check function, and
\(I(\cdot)\) is the indicator function.

Most commands for estimating quantile regression models focus on
estimating the above loss function, using linear programming techniques,
while others like \citet{kaplan2017} (\texttt{sivqr}) and
\citet{chernozhukov2022} (\texttt{qrprocess}) use other optimization
techniques.

When no unobserved heterogeneity is present, quantile regression model
can be easily implemented in a panel setting (see
\citet{wooldridge2010}), using a pooled version of the model. However,
when unobserved heterogeneity is present explicitly, the estimation of
quantile regressions is more challenging. Consider the case of panel
data such that the conditional quantile regression model is given by:

\begin{equation}\phantomsection\label{eq-feqr}{q_\tau(y_{it}|X_{it},1_i)=x_{it}\beta(\tau)+\alpha_i(\tau) 
}\end{equation}

This specification is explicitly considering that the unobserve effect
is identified for each \(i_{th}\) observation, and that this effect
varies across quantiles (\(\alpha_i(\tau)\)). A common approach, yet
incorrect due to the incidental parameter problem, is to estimate this
model by adding dummy variables for each individual in the quantile
regression model (as in \citet{budig2001}), or by demeaning the
explanatory variables (as in \citet{budig2010}). In fact, there is no
transformation of the data that can eliminate the individual fixed
effects, as it happens in standard linear regression models.

In this framework, the problem of the incidental parameter problem
occurs because the unobserved factors cannot be differenced out of. In
other words, unless specific assumptions are made, the estimation
requires the explicit estimation of the unobserved fixed effect.
Unfortunately, because the number of available observations per
individual fixed effect is limited, they cannot be estimated with
precision. In turn, the cumulative errors in the estimation of the fixed
effects will also affect the conditional distribution of the outcome,
which quantile regressions leverage on, leading to inconsistent
estimates of all parameters. \footnote{This is similar to the measuring
  error problem of dependent variables in quantile regression models
  discussed in \citet{hausman2021}.}

In the next section, we present a few solutions and implementations for
the estimation of quantile regression models with multiple fixed
effects.

\subsection{Correlated Random Effects: CRE}\label{sec-cre}

The first approach we discuss is the use of Correlated Random Effects
(CRE) models for the estimation of quantile regression models. The CRE
model is an alternative methodology for the estimation of fixed effects
models that was proposed by \citet{mundlak1978} and later generalized by
\citet{chamberlain1982}. In contrast with standard fixed effects, the
approach allows users to control for time fixed covariates in addition
to time-varying covariates. And, in contrast with the random effects
model, it does not make the assumption that the unobserved effect is
uncorrelated with the observed covariates. Interestingly, in the context
of linear models, the CRE model is equivalent to the fixed effects model
\citep{wooldridge2010}.

Consider the following model:

\begin{equation}\phantomsection\label{eq-cre-1}{y_{it} = x_{it}\beta + z_{i}\gamma + \alpha_i + u_{it}
}\end{equation}

It is well known that if \(\alpha_i\) is correlated with \(x_{it}\), the
Random Effects (RE) estimator will be inconsistent. This is similar to
the ommited variable bias. The solution proposed by \citet{mundlak1978}
and \citet{chamberlain1982} was to explicitly account for that
correlation in the model, by assuming the unobserved effect \(\alpha_i\)
is a linear projection of the observed time-varying variables.
Specifically:

\begin{equation}\phantomsection\label{eq-cre-2}{\begin{aligned}
Mundlack:  & & \alpha_i &= \gamma_0 + \bar x_{i}\gamma + v_i&  \\
Chamberlain: & & \alpha_i &= \gamma_0 + x_{i1}\gamma_1 + x_{i2}\gamma_2 + \dots + x_{iT}\gamma_T + v_i 
\end{aligned}
}\end{equation}

The main difference between both approaches was that Chamberlain's is
more flexible by allowing all realizations of the time-varying variables
to explain the unobserved effect. In contrast, Mundlak's approach only
considers the average of the time-varying variables, which is a more
restrictive specification. Using either model specification, if we
substitute Equation~\ref{eq-cre-2} into Equation~\ref{eq-cre-1}, the
final model can be written as:

\begin{equation}\phantomsection\label{eq-cre-final}{y_{it} = x_{it}\beta + z_{i}\gamma + \gamma_0 + f(x_{it})\Gamma + v_i + u_{it}
}\end{equation}

where \(f(x_{it})\) can be the full set of time-varying variables or
just the average of them. Interestingly, either method provides the same
results if the panel data is balanced, and all covariates are strictly
exogenous. However, this identity breaks down in other cases (see
\citet{abrevaya2013}).

The strategy proposed by \citet{abrevaya2008} was to extend the CRE
model (\citet{chamberlain1982} style) for the estimation of quantile
regression models. This, however, has some limitations. First, when the
number of periods is large, the number of additional regressors grows
quickly, which can lead to other problems during estimation. Second,
while the application of \citet{chamberlain1982} projection approach for
unbalance data is possible (see \citet{abrevaya2013}), it is not
straightforward to implement in practice, specially for the framework of
quantile regression. Instead, we follow \citet{wooldridge2010} and
\citet{wooldridge2019}, and use the Mundlak representation of the CRE
model for the estimation of quantile regression models.
\citet{wooldridge2019} has shown that this can be easily applied for
cases with unbalanced panels, and the estimation of non-linear models.

Specifically, \citet{wooldridge2010} suggests that we could estimate the
quantile regression model using the Mundlak representation of the CRE
model:

\[q_\tau(y_{it}|x_{it},\bar x_i)=x_{it}\beta(\tau)+\bar x_i\gamma(\tau) 
\]

One of the benefits from this approach is that it will not only allow to
easily use unbalanced panels, but may also provide an approach to
control for multiple fixed effects, as discussed in \citet{baltagi2023}.
For example, for a case with two dimensions of fixed effects (say
individual and time), we could model each time varying variable as
follows:

\[x^k_{it}-E(x_{it}^k)=\lambda^{xk}_i + \lambda^{xk}_t + \varepsilon_{it}
\]

Here, we use the centered transformation of the explanatory variable,
that is \(x^k_{it}-E(x_{it}^k)\), so that all \(\lambda's\) have an
expected value of zero. Also, \(\lambda^{xk}_i\) and \(\lambda^{xk}_t\)
are the equivalent to \(\bar x_i\) in the Mundlak one-way fixed effects
model, and they can be obtained using an iterative process similar to
\citet{rios2015} or \citet{correia_feasible_nodate}.\footnote{Internally,
  we use \citet{correia_feasible_nodate} \texttt{reghdfe} to obtain the
  predicted fixed effects} The final model can be written as:

\[q_\tau(y_{it}|x_{it},\lambda_i^x,\lambda_t^x)=x_{it}\beta(\tau)+\lambda^x_i\gamma_i(\tau)+\lambda^x_t\gamma_t(\tau)  
\]

Which could be extended to any number of fixed effects.

\subsection{Canay (2011) Estimator}\label{sec-canay}

The second approach under consideration is the estimator proposed by
\citet{canay2011}. This paper argues that the estimator proposed by
\citet{abrevaya2008}, and thus the implementation described above, may
not provide consistent estimates of the quantile regression
coefficients, as long as there is a disturbance \(\varepsilon_{it}\)
left after modeling Equation~\ref{eq-cre-2}. However, under the
assumption that the unobserved effect is a pure location shift, that is
\(\alpha_i\) does not vary across quantiles, they propose that a
two-step estimator can be used to consistently estimate the quantile
regression coefficients. The first one eliminating the unobserved fixed
effect, and the second one estimating the quantile coefficients.

To better understand how this estimator works, let us consider the
following data generating process (DGP):

\[y_{i} =\beta_0(\rho_{i}) +  \beta_1(\rho_{i}) x_{1i} + \beta_2(\rho_{i}) x_{2i} 
\]

This represents the random coefficient approach to quantile regression
models, where the coefficients of the model are allowed to vary as a
function of a random variable \(\rho_{i}\), which follows a uniform
distribution. \(\rho_i\) is unobserved, but it is assumed to be
independent of the covariates. Given this DGP, the conditional quantile
function can be written as:

\[q_{\tau}(y_{i}|x_{1i},x_{2i}) = \beta_0(\tau) + \beta_1(\tau) x_{1i} + \beta_2(\tau) x_{2i}\]

In this specification, the impact of \(x_{k}\) on \(y\) is allowed to
vary across quantiles. However, we could also impose the assumption that
some of the coefficients are constant across quantiles. For example, if
\(\beta_2(\tau)=\beta_2\) for all \(\tau\), we would be assuming that
\(x_2\) only has a location effect on \(y\). In fact, if all
coefficients (but \(\beta_0\)) are constant across quantiles, the model
could just as well be estimated using OLS. And under this scenario, it
may be convenient to estimate the model imposing this restriction in the
model. This is the main idea behind the Canay estimator.

The estimator proposed by \citet{canay2011} imposes the assumption that
the unobserved effect is a pure location shift. More explicit, assumes
that the data generating process in a panel data setting, can be written
as follows:
\[y_{it} =\beta_0(\rho_{it}) +  \beta_1(\rho_{it}) x_{it} + \alpha_i
\]

According to \citet{canay2011}, under the assumption that \(\alpha_i\)
is constant across quantiles, we could consistently estimate the
quantile regression coefficients \(\beta{\tau}\), by simply transforming
the dependent variable as follows:

\[y_{it}-\alpha_i =\tilde y_{it}=\beta_0(\rho_{it}) +  \beta_1(\rho_{it}) x_{it} 
\]

And then use standard quantile methods on \(\tilde{y}_{it}\). More
formally, he suggests the following two-step estimator:

\begin{enumerate}
\def\labelenumi{\arabic{enumi}.}
\tightlist
\item
  Using an OLS regression, run the following model:
  \[y_{it} = x_{it}\beta + \alpha_i + \varepsilon_{it}\]
\end{enumerate}

and esitmate \(\hat{\alpha}_i\).\footnote{Empirically, this can be done
  using \texttt{reghdfe} command.}

\begin{enumerate}
\def\labelenumi{\arabic{enumi}.}
\setcounter{enumi}{1}
\tightlist
\item
  Transform the dependent variable as
  \(\tilde y_{it}=y_{it}-\hat{\alpha}_i\), and estimate the quantile
  regression model:
\end{enumerate}

\[q_{\tau}(y_{it}-\hat \alpha_i|x_{it})=q_{\tau}(\tilde y_{it}|x_{it}) = x_{it}\beta(\tau)\]

This simple approach allows for the identification of the quantile
coefficients, by imposing the assumption that the unobserved effect is a
pure location shift. And like the CRE model, it can be extended to
multiple fixed effects, as long as one is willing to assume that the
unobserved effects are pure location shifts. For example, consider a
case with two fixed effects dimensions (individual and time), the model
could be written as:

\[y_{it} = x_{it}\beta + \alpha_i + \alpha_t + \varepsilon_{it}\]

As before \(\alpha_i\) and \(\alpha_t\), assumed constant across
quantiles, could be estimated using OLS, and the quantile regression
model estimated using the transformed dependent variable.

\[q_{\tau}(y_{it}-\hat \alpha_i - \hat \alpha_t|x_{it})=q_{\tau}(\tilde y_{it}|x_{it}) = x_{it}\beta(\tau)
\]

Which again can be easily estimated using standard quantile regression
methods.

\subsection{Modified Canay(2011) Estimator}\label{sec-mcanay}

Perhaps one of the main limitations of the Canay estimator is that it
assumes that the unobserved effect is a pure location shift. If this
assumption is violated, it may lead to inconsistent estimates of the
quantile coefficients. To address this limitation, we propose a small
modification to the Canay estimator.

We start by assuming that the unobserved effect represents some
characteristics of the individual that are constant across quantiles,
and that can be compared across individuals. Under this consideration,
the data generating process can be written as:

\[y_{it} =\beta_0(\rho_{it}) +  \beta_1(\rho_{it}) x_{it} + \gamma(\rho_{it}) \alpha_i
\]

Next, similar to \citet{canay2011}, we propose to estimate the
unobserved effect \(\alpha_i\) using OLS, and assume that, in average
the \(E[\gamma(\rho_{it})]=1\). Up to this point, the estimator is the
same as the Canay estimator. However, we propose that instead of
imposing \(\gamma(\rho_{it}) \alpha_i\) to be constant across quantiles,
we allow for the prediction of the unobserved effect \(\alpha_i\) to
vary across quantiles. This can be done by estimating the following
model:

\[q_{\tau}(y_{it}|x_{it},\hat \alpha_i) =  x_{it}\beta(\tau)+\gamma(\tau) \hat\alpha_i
\]

As before, this model can be extended to multiple fixed effects, by
simply estimating the unobserved effects using OLS, and then estimating
the quantile regression model using the predicted unobserved effects.
The main advantage over \citet{canay2011} is that this estimator allows
for the unobserved effect to have a different impact on the dependent
variable across quantiles, which may be more realistic in many
applications. However, it assumes the OLS estimator does allow for the
consistent estimation of an unobserved effect that is comprarable across
individuals, which may not always be the case.

\subsection{\texorpdfstring{Method of Moments Quantile Regression
\citet{mss2019}}{Method of Moments Quantile Regression @mss2019}}\label{sec-mmqr}

This approach of estimating regression quantiles with multiple fixed
effects distinguishes itself from the other estimators reviewed above in
the sense that \citep{mss2019} introduce location-scale effects of fixed
effects upon the distribution of interest. Compared to Canay's
estimator, the Method of Moments Quantile Regression (MMQREG) not only
allows the \(\alpha_i\)'s to affect \(Y_{it}\) through location shifts,
but rather MMQREG is able to identify the scale shifts that alter
different points of the distribution belonging to \(Y\).

We begin by defining the DGP of the location scale model:

\begin{equation}\phantomsection\label{eq-dgp_mmqreg}{Y_{it} = \alpha_i + X_{it}' \beta + (\delta_i + X_{it}' \gamma) u_{it}
}\end{equation}

Where parameters \(\alpha_i\) and \(\delta_i\) capture the individual
fixed effects. Note that, compared to equation
\hyperref[eq:dgp_canay]{{[}eq:dgp\_canay{]}}, the fixed effects not only
enter the model in an additive fashion, instead they also have a
multiplicative effect upon the error term. In addition, the \(U_{it}\)
are i.i.d. across \(i\) and \(t\), statistically independent of
\(X_{it}\) and satisfy \(E(U) = 0\) and \(E(|U|) = 1\) both of which
normalize the random variable.

Our location scale model in equation
\hyperref[eq:dgp_mmqreg]{{[}eq:dgp\_mmqreg{]}} implies that:

\[\label{eq:mmqreg_quantile}
    Q_{\tau}(Y_{it}|X_i) = [\alpha_i + \delta_i q(\tau)] + X_{it}' \beta + X_{it}' \gamma q(\tau)\]

Where the scalar coefficient
\(\alpha_i(\tau) \equiv \alpha_i + \delta_i q(\tau)\) is the
quantile-\(\tau\) fixed effect for individual \(i\), which represent how
time-invariant variables have \emph{different impacts on different
regions} of the conditional distribution of \(Y_{it}\). However, our
real interest is in the regression quantile coefficients:

\[\label{eq:rqcoefficients_mmqreg}
    \beta_{\tau} = \beta + q(\tau) \gamma\]

Which are simply a linear combination of the location coefficients
(\(\beta\)) and the scale coefficients (\(\gamma\)), where the second
vector of coefficients is weighted by the value of the \(\tau\)-th
quantile of the variable of interest \(Y_{it}\). \citep{mss2019} develop
the following algorithm for implementing the MMQREG estimator:

\begin{enumerate}
\def\labelenumi{\arabic{enumi}.}
\item
  Obtain \(\hat{\beta}_k\) by regressing time-demeaned \(Y_{it}\) on
  time-demeaned controls \(X_{it}\), i.e.~obtain \(\hat{\beta}\) by the
  within estimator.
\item
  Estimate the \(\hat{\alpha}_i\)'s and calculate the residuals
  \(\hat{R}_{it} = Y_{it} - \hat{\alpha}_i - X_{it}' \hat{beta}\).
\item
  Obtain \(\hat{\gamma}_k\) by the within estimator using
  \(|\hat{R}_{it}|\) as the dependent variable.
\item
  Estimate \(\hat{\delta}_i\) by taking the time average of
  \(|\hat{R}_{it}| - X_{it}' \hat{\gamma}\).
\item
  Estimate \(q(\tau)\) by \(\hat{q}\), which corresponds to the
  regression quantile of standardized residuals
  {[}\(\hat{R}_{it}/(\hat{\gamma}_{i} + X_{it}' \hat{\gamma})]\) upon an
  intercept term.
\end{enumerate}

Note that Steps 1 and 2 from the MMQREG algorithm are the same as those
performed in Canay's estimator. However, the location-scale model used
in \citep{mss2019} adds three additional steps that are required to
estimate \(\hat{\gamma}\) and \(\hat{q}\), so that regression quantile
coefficients \(\beta_{\tau}\) are allowed to affect not only the
location of the distribution, but also its shape.

Add notes on the assumptions of MMQREG here \emph{if} necessary.

Statistical inference can be performed using the asymptotic distribution
of the estimator derived in \citep{mss2019}. Expanding on this
literature, \citep{riosavila2024} propose methods for computing
alternative standard errors using the empirical influence functions of
the estimators. Robust and clustered standard errors can be estimated
following this approach, and are readily available as options in the
\texttt{mmqreg} command in Stata. Another extension of the MMQREG
estimator due to \citep{riosavila2024} is to allow for the inclusion of
multiple fixed effects using an application of the Frisch-Waugh-Lovell
(FWL) theorem to partial out the effect of variables capturing
unobserved heterogeneity from both dependent and explanatory variables.
Likewise, the command \texttt{mmqreg} allows for multiple fixed effects,
as will be shown next.

\begin{longtable}[]{@{}lll@{}}
\caption{Government spending and coalition government - MMQREG
Estimator}\tabularnewline
\toprule\noalign{}
Column 1 & Column 2 & Column 3 \\
\midrule\noalign{}
\endfirsthead
\toprule\noalign{}
Column 1 & Column 2 & Column 3 \\
\midrule\noalign{}
\endhead
\bottomrule\noalign{}
\endlastfoot
Data 1 & Data 2 & Data 3 \\
Data 4 & Data 5 & Data 6 \\
\end{longtable}

As we allow for both location and scale shifts in our estimation method,
the monotonic pattern observed in the first row of table
\hyperref[tab:mcanay_results]{3} is reversed. Now, the farther we move
from the median of the distribution in the direction of greater
government spending, the lower is the effect of a permanent change from
single-party to coalition upon spending. The median coefficient, once
again, remains close to the mean coefficient (2.36) under MMQREG
estimation. Results from table \hyperref[tab:mmqreg_results]{4} support
the conclusions from \citep{persson2007} regarding types of government
and its spending during the last year of legislature, expanding the
results to the response of the distribution of spending upon changes in
explanatory variables.

We must also note that statistical inference ---with robust standard
errors displayed in table \hyperref[tab:mmqreg_results]{4}--- for the
\citep{persson2007} dataset show that many coefficients that with the
previous estimators were not statistically different from zero, are now
significant even at the 99\% confidence level. This is due to the
ability to compute robust standard errors for the MMQREG estimator,
contrasting with the CRE and Canay estimators, for which we estimated
the covariance matrix using the bootstrap. Although not displayed here,
our findings are robust to clustering standard errors for country so
that the coefficients of 25th and 50th quantiles corresponding to the
indicator variable for coalition government remain statistically
significant.

\clearpage

\bibliographystyle{sj}
\bibliography{bibliography.bib}


\begin{aboutauthors}

Fernando Rios-Avila is a Research Scholar at the Levy Economics
Institute of Bard College.

Gustavo Canavire-Bacarreza is a Senior Economist at the World Bank.

\end{aboutauthors}

\end{document}
